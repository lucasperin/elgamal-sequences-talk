
\begin{frame}{Introduction}
    \begin{center}
        \emph{Sidon sets and statistics of the ElGamal function} \\
        \citet*{boppre2020sidon}
    \end{center}
    
    \begin{itemize}
        \item Started in 2016 as a research challenge by Joachim von zur Gathen;
        \item Boppré and Perin wrote a report with experimental analysis;
        \item By 2017, Ana and Joachim wrote the Sidon Set part and submitted to arxiv.
        \item In 2020, the paper was published in Cryptologia.
    \end{itemize}
\end{frame}


\begin{frame}{ElGamal Permutations}

    Let $G = \Z_p^\times = \{1, \ldots, p-1\}$ be a cyclic group of order $p-1$\\
    $p$ prime

    \pause
    \begin{itemize}
        \item ElGamal signatures uses the fact that $G = \{ g^x : x\in \Z_{p-1}\}$, where $g$ is a generator of $G$;
        \item $g^x$ is a unique representation of $x$, and thus it spans a permutation of $G$.
    \end{itemize}
    \pause
    We are interested on the randomness properties of the \emph{ElGamal} map from $\Z_{p-1}$ to $G$ with $b \to g^b $
    
    Lucas: USE BETTER NOTATION FROM PAPER HERE
\end{frame}

\begin{frame}{ElGamal Permutations}
Example: Let $p =5$, then $2$ and $3$ are generators of G = $\Z_p^\times$.

      \begin{columns}
        \begin{column}{0.45\textwidth}
        \centering
            \begin{table}[]
    	    \begin{tabular}{c|c}
    	        $x$ & $g^{x} $ \\ \hline \hline
    	        $1$ & $g^{1} = 2$ \\
    	        $2$ & $g^{2} = 4$ \\
    	        $3$ & $g^{3} = 3$ \\
    	        $4$ & $g^{4} = 1$  
    	    \end{tabular}
    	    \caption{$g^{x}$ with $x$ in $\mathbb{Z}_5^\times$ and $g=2$}
    	    \label{tab:xmap1}
    	    \end{table}
        \end{column}
        \begin{column}{0.45\textwidth}
    	    \centering
            \begin{table}[]
    	    \begin{tabular}{c|c}
    	        $x$ & $g^{x} $ \\ \hline \hline
    	        $1$ & $g^{1} = 3$ \\
    	        $2$ & $g^{2} = 4$ \\
    	        $3$ & $g^{3} = 2$ \\
    	        $4$ & $g^{4} = 1$
    	    \end{tabular}
    	    \caption{$g^{x^*}$ with $x$ in $\mathbb{Z}_5^\times$ and $g=3$}
    	    \label{tab:xmap2}
    	    \end{table}
        \end{column}
  \end{columns}
        \begin{columns}
        \begin{column}{0.45\textwidth}
        \centering
            cycles = \{\{1,2,4\},\{3\}\}
        \end{column}
        \begin{column}{0.45\textwidth}
    	    \centering
            cycles = \{1,2,3,4\}
        \end{column}
  \end{columns}
  
  \pause
  \begin{itemize}
      \item Distinct $g$ produce distinct permutations;
      \item Distinct $g$ affect the cyclic structures.
  \end{itemize}
  
\end{frame}


\begin{frame}{Pictorial Representation}
\end{frame}

\begin{frame}{Experimentation}
\end{frame}

\begin{frame}{Results with Sidon Sets}
\end{frame}

\begin{frame}{ElGamal Sequences}

An {\em ElGamal sequence} is obtained by reducing an ElGamal permutation modulo $v$:
  
  \[\gamma_v = ((g^0\rem p)\rem v,(g^1\rem p)\rem v,(g^2\rem p)\rem v,(g^3\rem p)\rem v,\ldots)\]

  How closely do these sequences compare to random balanced sequences over $\mathbb{Z}_v$?

\end{frame}

\begin{frame}{Randomness properties}
  \begin{itemize}
  \item Balance
  \item Period
  \item   $\lambda(z) = \cardinality{\{i \in [0,p-1]: \sigma(i+_{n} \iota) = z(\iota),\; 0 \leq \iota < t\}}$
    \item  $\rho(b,t) = \cardinality{ \{i \in [0,p-1]: \sigma(i-_{n} 1),\sigma(i+_n t) \neq b = \sigma(i+_n\iota),\; 0 \leq \iota < t\}}$
    \end{itemize}
  
  \end{frame}


\begin{frame}{ElGamal Run ratio Experiment}
    Show experiment with ratio against expected from Golomb's postulates
\end{frame}



 	

%%% Local Variables:
%%% TeX-master: "../main.tex"
%%% End:
